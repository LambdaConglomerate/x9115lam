\documentclass[english]{article}
\usepackage[T1]{fontenc}
\usepackage[latin9]{inputenc}
\usepackage{babel}
\begin{document}
\title{Paper Six Summary}


\author{Matthew Neal, Joseph Sankar, and Alexander Sobran}

\maketitle

\section*{Reference}

Rahman et al. \cite{Rahman} listed below.


\section*{Important Keywords}
\begin{description}
\item [{Cross-project defect prediction}] Using data from one project to predict defects in another.
\item [{Within-project defect prediction}] Using data from previous releases of a project to predict defects in future releases. For new projects, the lack of historical defect data makes this kind of defect prediction almost impossible.

\end{description}

\section*{Feature Extraction}
\begin{description}
\item[{Motivational Statements}]  While within-project defect prediction can be very effective, new projects don't have the volume of data needed to create these models. Cross-project defect prediction models aim to help with this issue, but so far the results have largely been disappointing. The authors hope to show that cross-project defect prediction can be roughly as effective as traditional defect prediction by using a different set of measures, namely those based on a variety of tradeoffs of time-and-cost vs. quality.

\end{description}


\section*{Possible Improvements}

\section*{Connection to Other Papers}

\bibliographystyle{plain}
\bibliography{references}
\end{document}