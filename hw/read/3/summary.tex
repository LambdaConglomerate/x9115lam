\documentclass[english]{article}
\usepackage[T1]{fontenc}
\usepackage[latin9]{inputenc}
\usepackage{babel}
\begin{document}

\title{Paper Three Summary}


\author{Matthew Neal, Joseph Sankar, and Alexander Sobran}

\maketitle

\section*{Reference}

Bird et al \cite{bird09} listed below.


\section*{Important Keywords}
\begin{description}
\item [{Global development}] Development that is globally distributed but takes place within one company. The authors contrast this to \emph{outsourcing}, in which development takes place among multiple companies and \emph{collocated development}, where development takes place in one location. 
\item [{Coordination breakdown}] A difficulty usually encountered in distributed development where employees in one location are not appropriately aware or responsive to the work of employees in another location.
\item [{Binary distribution}] Different levels of classifications for where binaries are developed. The authors give \emph{building, cafeteria, campus, locality, continent, } and \emph{world} as different ways to classify binary distribution.
\item [{Organizational integration}] An organizational structure that spans geographical locations at low levels.
\end{description}

\section*{Feature Extraction}
\begin{description}
\item[Hypotheses] The authors list two hypotheses \cite{bird09}: 
\begin{quote}
\emph{H1: Binaries that are developed by teams of engineers that are distributed will have more post-release failures than those developed by collocated engineers.}
\end{quote}
\begin{quote}
\emph{H2: Binaries that are distributed will be less complex, experience less code churn, and have fewer dependencies than collocated binaries.}
\end{quote}
\item[Future Work] The authors note that there haven't been any causal relationshisp established between development practices and effective distributed development. They suggest studies to examine the effects of various development practices on distributed development.
\end{description}

\section*{Possible Improvements}

\section*{Connection to Other Papers}
	Our first paper \cite{posnett11} cited geographical issues as a difficulty in interpreting aggregated results. They ask how the work should be divided among multiple locations to maximize quality and productivity, initially suggesting that development should not be split over different locations. They note that this study found different results, that geographical distribution had little to no effect on software quality. 

The authors also mentioned that this study examined software on the binary level (an aggregated level), which supports their claim that aggregation is very common in defect modeling research.


\bibliographystyle{plain}
\bibliography{references}
\end{document}
