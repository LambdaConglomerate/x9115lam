\documentclass[english]{article}
\usepackage[T1]{fontenc}
\usepackage[latin9]{inputenc}
\usepackage{babel}
\begin{document}
\title{Paper Nine Summary}


\author{Matthew Neal, Joseph Sankar, and Alexander Sobran}

\maketitle

\section*{Reference}

Nam et al. \cite{Nam} listed below.


\section*{Important Keywords}
\begin{description}
\item[{Transfer defect learning}] Extracting common knowledge from one task domain and transferring it to another. The transferred knowledge is then used to train a prediction model.
\item[{TCA+}] An improvement on Transfer Component Analysis. TCA can map the data of the source and target projects on a latent feature space, but is sensitive to normalization. TCA+ selects a proper normalization to yield better prediction performance.
\item[{Data set Characteristic Vector}] A vector of six elements each relating to the distance between pairs of instances of data. DCVs are used to see how similar two projects are (see below).
\item[{Similarity vector}] Represents the difference between two projects: a source and a target. Examples include "much more", "less", or "same". The values in the DCVs are used to calculate the similarity vectors.
\end{description}

\section*{Feature Extraction}


\section*{Possible Improvements}

\section*{Connection to Other Papers}

\bibliographystyle{plain}
\bibliography{references}
\end{document}