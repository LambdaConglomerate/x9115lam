\documentclass[english]{article}
\usepackage[T1]{fontenc}
\usepackage[latin9]{inputenc}
\usepackage{babel}
\begin{document}

\title{Paper Four Summary}


\author{Matthew Neal, Joseph Sankar, and Alexander Sobran}

\maketitle

\section*{Reference}

Vandecruys et al \cite{Vandecruys08} listed below.


\section*{Important Keywords}
\begin{description}
\item [{Ant Colony Optimization (ACO)}] A metaheuristic inspired by the ant behavior of determining shortest path by concentration of pheromone present on the path. Ants returning faster to the colony leave a greater concentration of pheromone by doubling over on their path which is used as an indicator of the shortest path for other ants. ACO's artificial ants iteratively and stochastically construct solutions and add pheromones to these solutions. The pheromone is defined as the number of ants recently choosing a solution. When a new artificial ant encounters a decision it is more likely to choose path with a greater concentration of pheromone. The concentration is adjusted based on the evaluation of the solution. An evaporation constant causes the pheromones to evaporate and lowers concentration for each path.
\end{description}

\section*{Feature Extraction}
\begin{description}
\item [{Motivational~statements}] The motivation of this paper is to apply AntMiner+, a rule-based Ant Colony Optimization classification technique to public repositories and assess it's prediction and inference capabilities against other previously successful classification techniques in the domain of software quality.
\end{description}



\section*{Possible Improvements}


\section*{Connection to Other Papers}



\bibliographystyle{plain}
\bibliography{references}
\end{document}