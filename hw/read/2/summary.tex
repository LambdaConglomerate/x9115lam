\documentclass[english]{article}
\usepackage[T1]{fontenc}
\usepackage[latin9]{inputenc}
\usepackage{babel}
\begin{document}

\title{Paper Two Summary}


\author{Matthew Neal, Joseph Sankar, and Alexander Sobran}

\maketitle

\section*{Reference}

Arisholm et al \cite{arisholm10} listed below.


\section*{Important Keywords}
\begin{description}
\item [{Fault-proneness}] The number of defects detected in a software-component.
A measurement of the likelihood of software or software component
error. 
\item [{Object-oriented~(OO)~code~measures}] Measurement of the structural
properties of the source code. 
\item [{Delta~measures}] Measurement of the amount of change (churn) in
a file between two successive releases. 
\item [{Process~measures}] Measures from a configuration management system.
Includes developer experience, the number of developers that have
edited a file, the number of faults in the previous releases, the
number of line added, the number of lines removed. 
\end{description}

\section*{Feature Extraction}
\begin{description}
\item [{Motivational~statements}] The authors mentioned that most previous
studies only examined the structure of the code when constructing
fault-proneness prediction models. In the few studies that did not,
they failed to systematically measure the cost-effectiveness of the
models with factors with different data collection costs. Very few
studies have looked at the methods and criteria for evaluating the
models. Finally, it is very difficult to draw specific conclusions
using confusion matrices since they do not provide any insight into
determining the cost-effectiveness of the models under test. 
\item [{Motivational Statements}] The motivation was to provide a systematic
methodology for comparing modeling techniques. The authors state that
little to no effort was put forth to prior to this paper in that area. 
\item [{Patterns}] The authors provide an overview of a number of different
techniques and different metric sets in an attempt to prove which
provides the best results across comparison techniques. The results
seem to show that the ROC curve combined with cost effectiveness comparisons
provide a good deal of insight into the performance of techniques
and matrices as compared to using the confusion matrix approach. 
\item [{Future~Work}] The authors used the default parameters for the
statistical techniques they employed in the study. They have stated
that they hope to tweak the parameters and observe the results to
see if there are optimizations that can be made while at the same
time preventing overfitting. The authors also stated they are going
to work on a large-scale evaluation of the costs and benefits of various
prediction models in the COS project. 
\item [{Informative Visualizations}] Figure four on page 12 of the paper
provides a really interesting visualization of the distribution of
cost efficiency of prediction models using Process and Object Oriented
metrics. The figure provides two lines, one for OO and one for Process
metrics, but they also provide clouds around those lines that demarcate
the 25th and 75th percentiles. The authors use this figure to show
the fact that the performance of OO metrics based models is poor compared
with Process. The baseline for this study was a line such that $y=x$
so as is shown in the figure, OO has very close to the baseline performance
while OO outperforms the baseline by a significant margin. 
\end{description}

\section*{Possible Improvements}
\begin{itemize}
\item In the Introduction section, the authors reference a paper published
in 1989 to cite a percentage of effort spent on testing. Is there
a more recent paper the authors could have chosen to cite? If not,
why quote a figure that is 26 years old and quite possibly outdated? 
\item While the way that the authors combined data in an attempt to consolidate
it to a smaller number of tables was very clever, it produces tables
that are very difficult to interpret. They have split the data into
a standard set of stats (mean, median, Q1, Q3) on half of the table
and then created a diagonal split on the other half of the table into
two sets of data, one for effect size and the other for the Wilcoxon
test data. It seems that they had so much data to report that it would
have been impossible to report it in the space they had available
without this type of table, but perhaps some further graphical representation
would have been preferable with an appendix of the tables and data
sets split out into more easily decipherable tables. 
\item The authors mentioned that one threat to the validity of the study
was that the cost of making the measures available and collecting
them was not accounted for. It seems that such calcuations should
be taken into account. Additionally, they note that the Process metric
set has a high cost of data reporting and collection but was found
to be the most cost-effective. It is hard to say if Process is indeed
the most cost-effective, since all cost measures were not taken into
consideration. A future study should include as many cost factors
as possible when determining the cost-effectiveness of a particular
method or metric.
\end{itemize}

\section*{Connection to Other Papers}

Posnett et al \cite{posnett11} cited this paper as well as Menzies
et al \cite{menzies10} as evidence that prediction models designed
for the aggregated level have poor performance at the disaggregated
level. But moreover the Posnett paper uses this paper as a methodological
foundation for the way that they went about using the ROC curve and
Cost Effectiveness analysis that they used in their study.

 \bibliographystyle{plain}
\bibliography{references}
 {} \bibliographystyle{plain} 
\end{document}
