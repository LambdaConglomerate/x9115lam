\documentclass[english]{article}
\usepackage[T1]{fontenc}
\usepackage[latin9]{inputenc}
\usepackage{babel}
\begin{document}

\title{Paper Two Summary}


\author{Matthew Neal, Joseph Sankar, and Alexander Sobran}

\maketitle

\section*{Reference}

Erik Arisholm, Lionel C. Briand, and Eivind B. Johannessen. 2010.
A systematic and comprehensive investigation of methods to build and
evaluate fault prediction models. J. Syst. Softw. 83, 1 (January 2010),
2-17. DOI=10.1016/j.jss.2009.06.055 http://dx.doi.org/10.1016/j.jss.2009.06.055


\section*{Important Keywords}
\begin{description}
\item [{Fault-proneness}] The number of defects detected in a software-component.
A measurement of the likelihood of software or software component
error. 
\item [{Object-oriented~(OO)~code~measures}] Measurement of the structural
properties of the source code. 
\item [{Delta~measures}] Measurement of the amount of change (churn) in
a file between two successive releases. 
\item [{Process~measures}] Measures from a configuration management system.
Includes developer experience, the number of developers that have
edited a file, the number of faults in the previous releases, the
number of line added, the number of lines removed.
\end{description}

\section*{Feature Extraction}
\begin{description}
\item [{Motivational~statements}] The authors mentioned that most previous
studies only examined the structure of the code when constructing
fault-proneness prediction models. In the few studies that did, they
did not systematically measure the cost-effectiveness of the models
with factors with different data collection costs. Very few studies
have looked at the methods and criteria for evaluating the models.
Finally, it is very difficult to draw general conclusions using confusion
matrices since they do not provide any insight into determining the
cost-effectiveness of the models under test.
\item [{Future~Work}] The authors used the default parameters for the
statistical techniques they employed in the study. They have stated
that they hope to tweak the parameters and observe the results to
see if there are optimizations that can be made while at the same
time try to avoid overfitting. The authors also stated they are going
to work on a large-scale evaluation of the costs and benefits of various
prediction models in the COS project.
\end{description}

\section*{Possible Improvements}
\begin{itemize}
\item In the Introduction section, the authors reference a paper published
in 1989 to cite a percentage of effort spent on testing. Is there
a more recent paper the authors could have chosen to cite? If not,
why quote a figure that is 26 years old and quite possibly outdated?\end{itemize}

\end{document}
