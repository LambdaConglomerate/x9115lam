% This is "sig-alternate.tex" V2.1 April 2013
% This file should be compiled with V2.5 of "sig-alternate.cls" May 2012
%
% This example file demonstrates the use of the 'sig-alternate.cls'
% V2.5 LaTeX2e document class file. It is for those submitting
% articles to ACM Conference Proceedings WHO DO NOT WISH TO
% STRICTLY ADHERE TO THE SIGS (PUBS-BOARD-ENDORSED) STYLE.
% The 'sig-alternate.cls' file will produce a similar-looking,
% albeit, 'tighter' paper resulting in, invariably, fewer pages.
%
% ----------------------------------------------------------------------------------------------------------------
% This .tex file (and associated .cls V2.5) produces:
%       1) The Permission Statement
%       2) The Conference (location) Info information
%       3) The Copyright Line with ACM data
%       4) NO page numbers
%
% as against the acm_proc_article-sp.cls file which
% DOES NOT produce 1) thru' 3) above.
%
% Using 'sig-alternate.cls' you have control, however, from within
% the source .tex file, over both the CopyrightYear
% (defaulted to 200X) and the ACM Copyright Data
% (defaulted to X-XXXXX-XX-X/XX/XX).
% e.g.
% \CopyrightYear{2007} will cause 2007 to appear in the copyright line.
% \crdata{0-12345-67-8/90/12} will cause 0-12345-67-8/90/12 to appear in the copyright line.
%
% ---------------------------------------------------------------------------------------------------------------
% This .tex source is an example which *does* use
% the .bib file (from which the .bbl file % is produced).
% REMEMBER HOWEVER: After having produced the .bbl file,
% and prior to final submission, you *NEED* to 'insert'
% your .bbl file into your source .tex file so as to provide
% ONE 'self-contained' source file.
%
% ================= IF YOU HAVE QUESTIONS =======================
% Questions regarding the SIGS styles, SIGS policies and
% procedures, Conferences etc. should be sent to
% Adrienne Griscti (griscti@acm.org)
%
% Technical questions _only_ to
% Gerald Murray (murray@hq.acm.org)
% ===============================================================
%
% For tracking purposes - this is V2.0 - May 2012

\documentclass{sig-alternate-05-2015}


\begin{document}

% Copyright
\setcopyright{acmcopyright}
%\setcopyright{acmlicensed}
%\setcopyright{rightsretained}
%\setcopyright{usgov}
%\setcopyright{usgovmixed}
%\setcopyright{cagov}
%\setcopyright{cagovmixed}


% DOI
\doi{10.475/123_4}

% ISBN
\isbn{123-4567-24-567/08/06}

%Conference
\conferenceinfo{PLDI '13}{June 16--19, 2013, Seattle, WA, USA}

\acmPrice{\$15.00}

%
% --- Author Metadata here ---
\conferenceinfo{WOODSTOCK}{'97 El Paso, Texas USA}
%\CopyrightYear{2007} % Allows default copyright year (20XX) to be over-ridden - IF NEED BE.
%\crdata{0-12345-67-8/90/01}  % Allows default copyright data (0-89791-88-6/97/05) to be over-ridden - IF NEED BE.
% --- End of Author Metadata ---

\title{Software Defect Prediction: Review, Commentary, and Future Work}
%
% You need the command \numberofauthors to handle the 'placement
% and alignment' of the authors beneath the title.
%
% For aesthetic reasons, we recommend 'three authors at a time'
% i.e. three 'name/affiliation blocks' be placed beneath the title.
%
% NOTE: You are NOT restricted in how many 'rows' of
% "name/affiliations" may appear. We just ask that you restrict
% the number of 'columns' to three.
%
% Because of the available 'opening page real-estate'
% we ask you to refrain from putting more than six authors
% (two rows with three columns) beneath the article title.
% More than six makes the first-page appear very cluttered indeed.
%
% Use the \alignauthor commands to handle the names
% and affiliations for an 'aesthetic maximum' of six authors.
% Add names, affiliations, addresses for
% the seventh etc. author(s) as the argument for the
% \additionalauthors command.
% These 'additional authors' will be output/set for you
% without further effort on your part as the last section in
% the body of your article BEFORE References or any Appendices.

\numberofauthors{3} %  in this sample file, there are a *total*
% of EIGHT authors. SIX appear on the 'first-page' (for formatting
% reasons) and the remaining two appear in the \additionalauthors section.
%
\author{
% You can go ahead and credit any number of authors here,
% e.g. one 'row of three' or two rows (consisting of one row of three
% and a second row of one, two or three).
%
% The command \alignauthor (no curly braces needed) should
% precede each author name, affiliation/snail-mail address and
% e-mail address. Additionally, tag each line of
% affiliation/address with \affaddr, and tag the
% e-mail address with \email.
%
% 1st. author
\alignauthor
Matthew Neal\\
       \affaddr{Department of Computer Science}\\
       \affaddr{North Carolina State University}\\
       \affaddr{Raleigh, North Carolina, USA}\\
       \email{meneal@ncsu.edu}
% 2nd. author
\alignauthor
Joseph Sankar\\
	   \affaddr{Department of Computer Science}\\
       \affaddr{North Carolina State University}\\
       \affaddr{Raleigh, North Carolina, USA}\\
       \email{jesankar@ncsu.edu}
% 3rd. author
\alignauthor Alexandar Sobran\\
       \affaddr{Department of Computer Science}\\
       \affaddr{North Carolina State University}\\
       \affaddr{Raleigh, North Carolina, USA}\\
       \email{aisobran@ncsu.edu}
\and  % use '\and' if you need 'another row' of author names
}
% There's nothing stopping you putting the seventh, eighth, etc.
% author on the opening page (as the 'third row') but we ask,
% for aesthetic reasons that you place these 'additional authors'
% in the \additional authors block, viz.
% Just remember to make sure that the TOTAL number of authors
% is the number that will appear on the first page PLUS the
% number that will appear in the \additionalauthors section.

\maketitle
\begin{abstract}
This paper provides a comprehensive review and commentary on software defect prediction research. Review and criticism of previous work is presented. Future steps in the domain are given.
\end{abstract}


%
% The code below should be generated by the tool at
% http://dl.acm.org/ccs.cfm
% Please copy and paste the code instead of the example below. 
%
\begin{CCSXML}
<ccs2012>
 <concept>
  <concept_id>10010520.10010553.10010562</concept_id>
  <concept_desc>Computer systems organization~Embedded systems</concept_desc>
  <concept_significance>500</concept_significance>
 </concept>
 <concept>
  <concept_id>10010520.10010575.10010755</concept_id>
  <concept_desc>Computer systems organization~Redundancy</concept_desc>
  <concept_significance>300</concept_significance>
 </concept>
 <concept>
  <concept_id>10010520.10010553.10010554</concept_id>
  <concept_desc>Computer systems organization~Robotics</concept_desc>
  <concept_significance>100</concept_significance>
 </concept>
 <concept>
  <concept_id>10003033.10003083.10003095</concept_id>
  <concept_desc>Networks~Network reliability</concept_desc>
  <concept_significance>100</concept_significance>
 </concept>
</ccs2012>  
\end{CCSXML}

\ccsdesc[500]{Computer systems organization~Embedded systems}
\ccsdesc[300]{Computer systems organization~Redundancy}
\ccsdesc{Computer systems organization~Robotics}
\ccsdesc[100]{Networks~Network reliability}


%
% End generated code
%

%
%  Use this command to print the description
%
%\printccsdesc

% We no longer use \terms command
%\terms{Theory}

\keywords{Fault prediction model, Software mining, Ant Colony Optimization, Classification, Defect Prediction}

\section{Introduction}
It is a well-known fact that software bugs are much cheaper and easier to fix before being released. But finding these bugs is often difficult and may not be cost effective to fix. Researchers have been working on fault detection models to predict which software modules are most likely to contain bugs post-release. Management can use these predictions to focus testing and bug-fixing efforts on those modules, resulting in fewer bugs in release which are less costly to fix.

In this paper, we focus on the advances made in software fault prediction models in the literature. Section 2 contains an assortment of existing work in the area.

\section{Related Work}
Cagatay Catal and Banu Diri \cite{Catal} gave an overview of software fault prediction studies and advancements up to 2008. They found that an increasing number of studies used datasets available to the public. They also found that since 2005, machine learning algorithms have become increasingly popular choices to implement the models. Finally, they observed that the most dominant metrics in fault prediction were at the method level. They recommended that machine learning algorithms and public datasets continue to be used, but caution against using method-level metrics and instead suggested class-level metrics as they can predict faults earlier in the software development cycle.

Vandecruys et al \cite{Vandecruys} mined software repositories to create predictive models. The authors used AntMiner+, an Ant Colony Optimization (ACO)-based classification technique. On public datasets, AntMiner+ was found to be competitive to alternative classification techniques, such as C4.5, logistic regression, and support vector machines. The authors suggested that software managers would find the output of rules produced by AntMiner+ easy to understand and accept.

While there have been plenty of studies about predicting software faults based on the code itself, few studies have looked at organizational structure as a factor. Nagappan et al \cite{Nagappan} used organizational metrics, such as number of engineers, edit frequency, and organizational intersection factor to predict fault-proneness. The authors compared the effectiveness of the resulting model with alternative models which use traditional software metrics, like code churn and code complexity. They found that the model derived from organizational metrics had better precision and recall than others derived from software metrics, meaning they can also be effective indicators of failure-proneness.

Bird et al \cite{Bird} examined whether development that is largely distributed produces more failures than development that is mainly collocated. The belief at the time was that global development was prone to more failures than collocated development. They found a negligible difference in both code metrics and failures between the two methods of development. The authors examined how the developers working on Windows Vista managed to work well together among teams located in different countries based on the relationships between the development sites, cultural barriers, communication, consistent use of tools, end to end ownership, common schedules, and organizational integration. They recommended that companies wishing to distribute development across sites located far apart to employ similar strategies to overcome some of the difficulties associated with such an endeavor. 

Arisholm et al \cite{Arisholm} examined different ways to build and evaluate fault prediction models. First, they tested a variety of modeling techniques, such as neural networks, C4.5 with some variants, support vector machines, and logistic regression. They then looked at different metrics:  Process measures, OO code measures, and delta measures. Finally, they looked at ROC area and cost-effectiveness as evaluation criteria. They found that the choice of modeling technique did not have much of an impact when evaluated with both criteria tested. Among the metric sets, they found that process measures provide a significant improvement over the others, even though they are typically more costly to collect.  They also suggested cost-effectiveness as a good evaluation technique instead of traditional techniques like precision and recall as smarter decisions can be made to prioritize which parts of the project are tested and bug-fixed.

Jun Zheng \cite{zheng2010} argued that the mistakenly predicting a module as non-defective is more dangerous and costly than mistakenly predicting a module as defective. He presented three algorithms which boost neural networks to predict software defects with cost in mind. When evaluated on four NASA datasets, he found that threshold moving algorithm gave the best results in the sole Normalized Expected Cost of Misclassification (NECM) measure. He especially recommended threshold-moving algorithms on projects written in object oriented languages.

Most software products are structured in a hierarchical format, for example into methods, classes, files, packages, etc. What level of analysis should fault prediction models operate on? And can analysis on one level of study apply to other levels? Posnett et al \cite{Posnett} examines these issues. They observed that sometimes relevant phenomena only occur at an aggregated level or data may only be observed at an aggregated level, meaning that studies are often conducted at aggregated levels. They also noted that in other fields of study there are ecological fallacies which make findings at aggregated levels not apply at disaggregated levels. They found that much of these fallacies also exist in the domain of computer software and that care needs to be taken when employing ecological inference. As to how exactly the risks of ecological fallacies can be dealt with, the authors left open for future research.

Traditional software fault prediction models are trained on historical project data. Since new projects do not have such a volume of historical data, these fault prediction models are not as effective at within-project defect prediction. A possible solution to this issue is cross-project prediction which uses data from one project to predict defects in another. Still, models using this technique exhibit poor performance. Rahman et al \cite{Rahman} argued that one reason for such behavior is that standard evaluation measures such as precision, recall, and F-measure are taken at specific threshold settings, while they really should be taken in a range of time/cost vs. quality tradeoffs. They took the standard measures at a variety of tradeoffs and found that cross-project defect prediction is at least as good as within-project defect prediction, and sometimes substantially better.

Also within the area of cross-project defect prediction, Nam et al \cite{Nam} found that when the source and target projects have a different feature distributions, the resulting model gives poor performance. The authors use Transfer Component Analysis (TCA) to find a common feature set for both projects and then map the data of both projects to it. They found that TCA is sensitive to normalization, so they developed an improvement, TCA+ to select appropriate normalization options. After using TCA+ to create cross-project defect prediction models for eight open-source projects, they found a significant improvement in prediction performance compared to traditional algorithms and techniques. The authors proposed applying knowledge in one domain to another to further improve performance. 

As another approach to improving fault prediction, Tian Jiang \cite{Jiang} introduced \emph{personalized defect prediction}. He argued that developers have different coding styles and techniques and that if each developer had their own defect prediction model, performance would improve. He was careful to note that the developer was not a feature of the model, but that there was a model for each developer. He ran experiments on six large open-source projects using PCC+, a model which chooses the highest confidence prediction among CC (traditional change classification), PCC (personalized change classification), and weighted PCC (PCC with changes from other developers added to a developer's model). Compared to CC and MARS (another predictor which also creates different models for different groups of data), PCC+ outperformed CC, MARS, and PCC as long as there is enough training data for each developer. Jiang recommended that PCC+ be applied to other recommendation systems and types of predictions.

\section{Machine Learning Algorithms}

\section{Metrics}

\section{Features}

\section{Data Sources}
The choice of dataset is extremely important, especially when doing software defect prediction research. Datasets should be representative of actual software projects and should ideally be accessible to the public so the results can be re-tested and verified by others. Within the referenced papers, we found a variety of data sources which we feel need to be discussed. Primarily, we found open-source datasets, NASA datasets, and Windows Vista datasets. We will discuss each along with some reasons why they may have been chosen.
\subsection{Open-source}
\subsection{NASA}
A few papers (Zheng \cite{zheng2010}...) used datasets from NASA. Among them were KC1, used for storage management, KC2, used for scientific data processing, CM1, used for NASA spacecraft instruments, and PC1, used for flight software.

One major possibility for using NASA datasets is that most of their software is critical. A malfunction or crash due to a software defect is not only costly, but can also be deadly. NASA's software needs to be thoroughly tested and bug-fixed before it is ready for use. NASA's datasets contain code metrics and defect information, which is easy to mine and create training and test data from.

NASA datasets are also publicly available. Researchers can more easily verify the results obtained in a study. Also, since these datasets are static, they can be used as a baseline to compare the results from a newly created model against those from other studies.
\subsection{Windows Vista}

\section{Data Preprocessing}

\section{Cross-Project Defect Prediction}

\subsection{Context Factors}
Context factors for projects were assessed by Fend at el (2014 cite here). Used as a measurement of project similarity. The provides a set of measurements to rank a project's relatedness when performing cross-project defect prediction.

\textbf{Programming Languages} Projects can be divided into groups depending on the programming language they are written in. Projects written in multiple languages could be problematic unless multiple language categories are included.

\textbf{Issue Tracking} Whether or not a project uses an issue tracking system.

\textbf{Total Lines Of Code} Project size based on the total number of lines of code split by quartiles.

\textbf{Total Number of Commits}Project size as a function of the number of commits. Split into quartiles.

\textbf{Total Number of Developers} Project size as a determined by the number of developer. Split into quartiles.

\subsection{Ranking}

\subsection{Clustering}



\subsection{Math Equations}
You may want to display math equations in three distinct styles:
inline, numbered or non-numbered display.  Each of
the three are discussed in the next sections.

\subsubsection{Inline (In-text) Equations}
A formula that appears in the running text is called an
inline or in-text formula.  It is produced by the
\textbf{math} environment, which can be
invoked with the usual \texttt{{\char'134}begin. . .{\char'134}end}
construction or with the short form \texttt{\$. . .\$}. You
can use any of the symbols and structures,
from $\alpha$ to $\omega$, available in
\LaTeX\cite{Lamport:LaTeX}; this section will simply show a
few examples of in-text equations in context. Notice how
this equation: \begin{math}\lim_{n\rightarrow \infty}x=0\end{math},
set here in in-line math style, looks slightly different when
set in display style.  (See next section).

\subsubsection{Display Equations}
A numbered display equation -- one set off by vertical space
from the text and centered horizontally -- is produced
by the \textbf{equation} environment. An unnumbered display
equation is produced by the \textbf{displaymath} environment.

Again, in either environment, you can use any of the symbols
and structures available in \LaTeX; this section will just
give a couple of examples of display equations in context.
First, consider the equation, shown as an inline equation above:
\begin{equation}\lim_{n\rightarrow \infty}x=0\end{equation}
Notice how it is formatted somewhat differently in
the \textbf{displaymath}
environment.  Now, we'll enter an unnumbered equation:
\begin{displaymath}\sum_{i=0}^{\infty} x + 1\end{displaymath}
and follow it with another numbered equation:
\begin{equation}\sum_{i=0}^{\infty}x_i=\int_{0}^{\pi+2} f\end{equation}
just to demonstrate \LaTeX's able handling of numbering.

\subsection{Citations}
Citations to articles \cite{bowman:reasoning,
clark:pct, braams:babel, herlihy:methodology},
conference proceedings \cite{clark:pct} or
books \cite{salas:calculus, Lamport:LaTeX} listed
in the Bibliography section of your
article will occur throughout the text of your article.
You should use BibTeX to automatically produce this bibliography;
you simply need to insert one of several citation commands with
a key of the item cited in the proper location in
the \texttt{.tex} file \cite{Lamport:LaTeX}.
The key is a short reference you invent to uniquely
identify each work; in this sample document, the key is
the first author's surname and a
word from the title.  This identifying key is included
with each item in the \texttt{.bib} file for your article.

The details of the construction of the \texttt{.bib} file
are beyond the scope of this sample document, but more
information can be found in the \textit{Author's Guide},
and exhaustive details in the \textit{\LaTeX\ User's
Guide}\cite{Lamport:LaTeX}.

This article shows only the plainest form
of the citation command, using \texttt{{\char'134}cite}.
This is what is stipulated in the SIGS style specifications.
No other citation format is endorsed or supported.

\subsection{Tables}
Because tables cannot be split across pages, the best
placement for them is typically the top of the page
nearest their initial cite.  To
ensure this proper ``floating'' placement of tables, use the
environment \textbf{table} to enclose the table's contents and
the table caption.  The contents of the table itself must go
in the \textbf{tabular} environment, to
be aligned properly in rows and columns, with the desired
horizontal and vertical rules.  Again, detailed instructions
on \textbf{tabular} material
is found in the \textit{\LaTeX\ User's Guide}.

Immediately following this sentence is the point at which
Table 1 is included in the input file; compare the
placement of the table here with the table in the printed
dvi output of this document.

\begin{table}
\centering
\caption{Frequency of Special Characters}
\begin{tabular}{|c|c|l|} \hline
Non-English or Math&Frequency&Comments\\ \hline
\O & 1 in 1,000& For Swedish names\\ \hline
$\pi$ & 1 in 5& Common in math\\ \hline
\$ & 4 in 5 & Used in business\\ \hline
$\Psi^2_1$ & 1 in 40,000& Unexplained usage\\
\hline\end{tabular}
\end{table}

To set a wider table, which takes up the whole width of
the page's live area, use the environment
\textbf{table*} to enclose the table's contents and
the table caption.  As with a single-column table, this wide
table will ``float" to a location deemed more desirable.
Immediately following this sentence is the point at which
Table 2 is included in the input file; again, it is
instructive to compare the placement of the
table here with the table in the printed dvi
output of this document.


\begin{table*}
\centering
\caption{Some Typical Commands}
\begin{tabular}{|c|c|l|} \hline
Command&A Number&Comments\\ \hline
\texttt{{\char'134}alignauthor} & 100& Author alignment\\ \hline
\texttt{{\char'134}numberofauthors}& 200& Author enumeration\\ \hline
\texttt{{\char'134}table}& 300 & For tables\\ \hline
\texttt{{\char'134}table*}& 400& For wider tables\\ \hline\end{tabular}
\end{table*}
% end the environment with {table*}, NOTE not {table}!

\subsection{Figures}
Like tables, figures cannot be split across pages; the
best placement for them
is typically the top or the bottom of the page nearest
their initial cite.  To ensure this proper ``floating'' placement
of figures, use the environment
\textbf{figure} to enclose the figure and its caption.

This sample document contains examples of \textbf{.eps} files to be
displayable with \LaTeX.  If you work with pdf\LaTeX, use files in the
\textbf{.pdf} format.  Note that most modern \TeX\ system will convert
\textbf{.eps} to \textbf{.pdf} for you on the fly.  More details on
each of these is found in the \textit{Author's Guide}.

\begin{figure}
\centering
\caption{A sample black and white graphic.}
\end{figure}

\begin{figure}
\centering
\caption{A sample black and white graphic
that has been resized with the \texttt{includegraphics} command.}
\end{figure}


As was the case with tables, you may want a figure
that spans two columns.  To do this, and still to
ensure proper ``floating'' placement of tables, use the environment
\textbf{figure*} to enclose the figure and its caption.
and don't forget to end the environment with
{figure*}, not {figure}!

\begin{figure*}
\centering
\caption{A sample black and white graphic
that needs to span two columns of text.}
\end{figure*}


\begin{figure}
\centering
\caption{A sample black and white graphic that has
been resized with the \texttt{includegraphics} command.}
\vskip -6pt
\end{figure}

\subsection{Theorem-like Constructs}
Other common constructs that may occur in your article are
the forms for logical constructs like theorems, axioms,
corollaries and proofs.  There are
two forms, one produced by the
command \texttt{{\char'134}newtheorem} and the
other by the command \texttt{{\char'134}newdef}; perhaps
the clearest and easiest way to distinguish them is
to compare the two in the output of this sample document:

This uses the \textbf{theorem} environment, created by
the\linebreak\texttt{{\char'134}newtheorem} command:
\newtheorem{theorem}{Theorem}
\begin{theorem}
Let $f$ be continuous on $[a,b]$.  If $G$ is
an antiderivative for $f$ on $[a,b]$, then
\begin{displaymath}\int^b_af(t)dt = G(b) - G(a).\end{displaymath}
\end{theorem}

The other uses the \textbf{definition} environment, created
by the \texttt{{\char'134}newdef} command:
\newdef{definition}{Definition}
\begin{definition}
If $z$ is irrational, then by $e^z$ we mean the
unique number which has
logarithm $z$: \begin{displaymath}{\log e^z = z}\end{displaymath}
\end{definition}

Two lists of constructs that use one of these
forms is given in the
\textit{Author's  Guidelines}.
 
There is one other similar construct environment, which is
already set up
for you; i.e. you must \textit{not} use
a \texttt{{\char'134}newdef} command to
create it: the \textbf{proof} environment.  Here
is a example of its use:
\begin{proof}
Suppose on the contrary there exists a real number $L$ such that
\begin{displaymath}
\lim_{x\rightarrow\infty} \frac{f(x)}{g(x)} = L.
\end{displaymath}
Then
\begin{displaymath}
l=\lim_{x\rightarrow c} f(x)
= \lim_{x\rightarrow c}
\left[ g{x} \cdot \frac{f(x)}{g(x)} \right ]
= \lim_{x\rightarrow c} g(x) \cdot \lim_{x\rightarrow c}
\frac{f(x)}{g(x)} = 0\cdot L = 0,
\end{displaymath}
which contradicts our assumption that $l\neq 0$.
\end{proof}

Complete rules about using these environments and using the
two different creation commands are in the
\textit{Author's Guide}; please consult it for more
detailed instructions.  If you need to use another construct,
not listed therein, which you want to have the same
formatting as the Theorem
or the Definition shown above,
use the \texttt{{\char'134}newtheorem} or the
\texttt{{\char'134}newdef} command,
respectively, to create it.

\subsection*{A {\secit Caveat} for the \TeX\ Expert}
Because you have just been given permission to
use the \texttt{{\char'134}newdef} command to create a
new form, you might think you can
use \TeX's \texttt{{\char'134}def} to create a
new command: \textit{Please refrain from doing this!}
Remember that your \LaTeX\ source code is primarily intended
to create camera-ready copy, but may be converted
to other forms -- e.g. HTML. If you inadvertently omit
some or all of the \texttt{{\char'134}def}s recompilation will
be, to say the least, problematic.

\section{Conclusions}
This paragraph will end the body of this sample document.
Remember that you might still have Acknowledgments or
Appendices; brief samples of these
follow.  There is still the Bibliography to deal with; and
we will make a disclaimer about that here: with the exception
of the reference to the \LaTeX\ book, the citations in
this paper are to articles which have nothing to
do with the present subject and are used as
examples only.
%\end{document}  % This is where a 'short' article might terminate

%ACKNOWLEDGMENTS are optional
\section{Acknowledgments}
This section is optional; it is a location for you
to acknowledge grants, funding, editing assistance and
what have you.  In the present case, for example, the
authors would like to thank Gerald Murray of ACM for
his help in codifying this \textit{Author's Guide}
and the \textbf{.cls} and \textbf{.tex} files that it describes.

%
% The following two commands are all you need in the
% initial runs of your .tex file to
% produce the bibliography for the citations in your paper.
\bibliographystyle{abbrv}
\bibliography{references}  % sigproc.bib is the name of the Bibliography in this case
% You must have a proper ".bib" file
%  and remember to run:
% latex bibtex latex latex
% to resolve all references
%
% ACM needs 'a single self-contained file'!
%
%APPENDICES are optional
%\balancecolumns
\appendix
%Appendix A
\section{Headings in Appendices}
The rules about hierarchical headings discussed above for
the body of the article are different in the appendices.
In the \textbf{appendix} environment, the command
\textbf{section} is used to
indicate the start of each Appendix, with alphabetic order
designation (i.e. the first is A, the second B, etc.) and
a title (if you include one).  So, if you need
hierarchical structure
\textit{within} an Appendix, start with \textbf{subsection} as the
highest level. Here is an outline of the body of this
document in Appendix-appropriate form:
\subsection{Introduction}
\subsection{The Body of the Paper}
\subsubsection{Type Changes and  Special Characters}
\subsubsection{Math Equations}
\paragraph{Inline (In-text) Equations}
\paragraph{Display Equations}
\subsubsection{Citations}
\subsubsection{Tables}
\subsubsection{Figures}
\subsubsection{Theorem-like Constructs}
\subsubsection*{A Caveat for the \TeX\ Expert}
\subsection{Conclusions}
\subsection{Acknowledgments}
\subsection{Additional Authors}
This section is inserted by \LaTeX; you do not insert it.
You just add the names and information in the
\texttt{{\char'134}additionalauthors} command at the start
of the document.
\subsection{References}
Generated by bibtex from your ~.bib file.  Run latex,
then bibtex, then latex twice (to resolve references)
to create the ~.bbl file.  Insert that ~.bbl file into
the .tex source file and comment out
the command \texttt{{\char'134}thebibliography}.
% This next section command marks the start of
% Appendix B, and does not continue the present hierarchy
\section{More Help for the Hardy}
The sig-alternate.cls file itself is chock-full of succinct
and helpful comments.  If you consider yourself a moderately
experienced to expert user of \LaTeX, you may find reading
it useful but please remember not to change it.
%\balancecolumns % GM June 2007
% That's all folks!
\end{document}
