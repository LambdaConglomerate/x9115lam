%% LyX 2.1.4 created this file.  For more info, see http://www.lyx.org/.
%% Do not edit unless you really know what you are doing.
\documentclass[english]{article}
\usepackage[T1]{fontenc}
\usepackage[latin9]{inputenc}
\usepackage{babel}
\begin{document}

\title{Paper One Summary}


\author{Matthew Neal, Joseph Sankar, and Sasha Sobran}

\maketitle

\section*{Reference}

Daryl Posnett, Vladimir Filkov, and Premkumar Devanbu. 2011. Ecological
Inference in Empirical Software Engineering. Proceedings of the 2011
26th IEEE/ACM International Conference on Automated Software Engineering. 


\section*{Important Keywords}
\begin{description}
\item [{Ecological~inference}] The idea that an observation or finding
at an aggregated level can apply at a disaggregated level. For example,
the idea that a model of defects at the package level also applies.
at the file level
\item [{Ecological~fallacy}] An incorrect application of ecological inference. A discrepency between the findings at an aggregated level and disaggragated level.
For example, an instance where a model of defects at the product level
is incorrectly used to model defects at the file level.
\item [{Scale}] Size of an aggregated unit. The larger the scale the bigger and fewer the aggregated units. This can effect affect the quality of statistical models built for defect prediction.
\item [{Zonation}] The manner in which aggregation is performed. For example, gerrymandering. This can create internal validity threats and lead to ecological fallacies.
\end{description}

\section*{Feature Extraction}
\begin{description}
\item [{1~-~Motivational~Statements}] Modern software applications are
composed of many files. Thus, the files are usually organized into
units, which may themselves be organized into broader units. While
research could be focused at the most granular level, for example
files, methods, even lines of code, research at these levels is costly
and takes time, so research instead usually examines data at higher
levels. Therefore, it would be useful to draw conclusions from higher
levels that would apply to lower levels.
\end{description}

\section*{Possible Improvements}
\end{document}
