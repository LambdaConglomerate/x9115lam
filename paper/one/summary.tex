%% LyX 2.1.4 created this file.  For more info, see http://www.lyx.org/.
%% Do not edit unless you really know what you are doing.
\documentclass[english]{article}
\usepackage[T1]{fontenc}
\usepackage[latin9]{inputenc}
\usepackage{babel}
\begin{document}

\title{Paper One Summary}


\author{Matthew Neal, Joseph Sankar, and Alexander Sobran}

\maketitle

\section*{Reference}

Daryl Posnett, Vladimir Filkov, and Premkumar Devanbu. 2011. Ecological
Inference in Empirical Software Engineering. Proceedings of the 2011
26th IEEE/ACM International Conference on Automated Software Engineering. 


\section*{Important Keywords}
\begin{description}
\item [{Ecological~inference}] The idea that an observation or finding
at an aggregated level can apply at a disaggregated level. For example,
the idea that a model of defects at the package level also applies.
at the file level
\item [{Ecological~fallacy}] An incorrect application of ecological inference. A discrepency between the findings at an aggregated level and disaggragated level.
For example, an instance where a model of defects at the product level
is incorrectly used to model defects at the file level.
\item [{Scale}] Size of an aggregated unit. The larger the scale the bigger and fewer the aggregated units. This can effect affect the quality of statistical models built for defect prediction.
\item [{Zonation}] The manner in which aggregation is performed. For example, gerrymandering. This can create internal validity threats and lead to ecological fallacies.
\item [{Aggregate~Phenomena}] Certain phenomena may apply only at an aggregated level.  The example given in the paper is that of inheritance.  Inheritance only takes place at the class level and cannot be studied at a disaggregated level.  
\item [{Cost~Effectiveness}] A concept used within the paper to assess defect models.  The baseline is that funds  are given to investigate {10\%} of a project for defects, in {10\%} of the lines of code taken at random {10\%} of the defects should be found.  Models are compared against this baseline.  
\end{description}

\section*{Feature Extraction}
\begin{itemize}
\item \textbf{Motivational Statements} Modern software applications are
composed of many files. Thus, the files are usually organized into
units, which may themselves be organized into broader units. While
research could be focused at the most granular level, for example
files, methods, even lines of code, research at these levels is costly
and takes time, so research instead usually examines data at higher
levels. Therefore, it would be useful to draw conclusions from higher
levels that would apply to lower levels.
\item \textbf{Related Work} To improve model performance, Koru \textit{et al.} recommend aggregating. Their methods to acheive improved performance include summing method-level measures into class-level measures and combining these with metrics from class-level features. They conclude that this approach overcomes the issues of asymmetry in defect data sets. Schr\"{o}ter \textit{et al.} conclude that predicting at coarser granularities is less difficult in comparison to finer granularities but lead to worse results. Zimmerman \textit{et al.} find different results where package level correlations and model correlations are higher than file level correlations and model correlations. Ambrose \textit{et al.} argue against the use of aggregate level metrics, though without data, stating predictions are more cumbersome to investigate,classes are inherently self-contained, and information derived from the package-level cannot be applied to the class level.
\item \textbf{Motivational Statements}: The authors poses their motivation along with the central question of the paper "What is the right level of study?" (At an aggregated level or dis-aggregated level).  They believe that this question has a central significance to whether or not a model that is built at a specific level of aggregation can hold at a lower level of aggregation as well as whether performance of the model as regards cost prediction is valid.
\item \textbf{New Results}:  The authors grouped variables in the models based on loss of significance(LS) or gain of significance(GS).  They explain that both LS and GS can cause Ecological Inferences issues as in a loss of significance situation when inferring between the aggregate level and the disaggregate level  or GS when inferring from the disaggregate to the aggregate.  They found 28 instances of GS and 25 instances of LS out of 108 variables over 68 models.  The authors draw the conclusion that models built at the aggregated level should be undertaken with the knowledge of the underlying risk of Ecological Inference and Ecological Fallacy.
\item \textbf{Future Work}:The authors bring up three sources of Ecological Inference risk:  Zonation, Sample Size, and Class Imbalance.  The authors believe that further study of the three sources is in order.  To give a short background on how each applies:
\begin{itemize}
\item Zonation: Aggregation runs the risk of creating distortion by the way that the data is aggregated.  
\item Sample Size:  As aggregation takes place inherently there is move to smaller and smaller sample sizes.  As the move to those smaller sample sizes takes place there is a loss in Statistical power.
\item Class Imbalance: A situation where a set of samples have a particular attribute and a large proportion of that set have a particular value with a very small number having another significant but underrepresented value.  The example in the paper is a situation where there are a very high number of projects without defects and a very lower number with defects.
\end{itemize}
\item \textbf{Sampling Procedures}: Projects were taken from a set of 18 Apache Software Foundation Projects over 87 versions.  The writers took JIRA issue reports and then mapped them to pull commits.  Then, they took those commits and connected them to packages and files.  They also pulled the number of lines of code as well as the number of developers per project.  They were unable to look at past data sets due to their inability to correctly aggregate the data given that many data sets do not contain issue identifiers.

\end{itemize}

\section*{Possible Improvements}
\begin{itemize}
\item There are some particularly distracting typos in the paper.  For example, on pages 367-368 the authors failed to correctly reference their table correctly leaving "Table ??" in the middle of their text twice.  
\end{itemize}
\end{document}
